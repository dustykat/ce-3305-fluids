\documentclass[12pt]{article}
\usepackage{geometry}                % See geometry.pdf to learn the layout options. There are lots.
\geometry{letterpaper}                   % ... or a4paper or a5paper or ... 
%\geometry{landscape}                % Activate for for rotated page geometry
\usepackage[parfill]{parskip}    % Activate to begin paragraphs with an empty line rather than an indent
\usepackage{daves,fancyhdr,natbib,graphicx,dcolumn,amsmath,lastpage,url}
\usepackage{amsmath,amssymb,epstopdf,longtable}
\usepackage{paralist}  % need to properly formulate standard answer blocks
\usepackage[final]{pdfpages}
\DeclareGraphicsRule{.tif}{png}{.png}{`convert #1 `dirname #1`/`basename #1 .tif`.png}
\pagestyle{fancy}
\lhead{CE 3305 Fluid Mechanics; Exercise Set 12}
\rhead{Name:\_\_\_\_\_\_\_\_\_\_\_\_\_\_\_\_\_\_\_\_\_\_\_\_\_\_\_\_\_\_\_\_\_\_}
\lfoot{REVISION A}
\cfoot{}
\rfoot{Page \thepage\ of \pageref{LastPage}}
\renewcommand\headrulewidth{0pt}
%%%%%%%%%%%%%%%%%%%%%%%%%%%%%%%%%%%%
\begin{document}
%%%%%%%%%%%%%%%%%%%%%%%%%%%%%%%%%%%
\begingroup
\begin{center}
{\textbf{{ CE 3305 Engineering Fluid Mechanics} \\ Exercise Set 12 (Dimensional Analysis \& Similitude) \\ Summer 2018 -- GERMANY} }
\end{center}
\endgroup
\begingroup
~\newline
\textbf{Purpose} :  Similitude Relationships \\
\textbf{Assessment Criteria} : Completion, plausible solutions, use \textbf{R} as a calculator. \\~\\
\textbf{Exercises}

\begin{enumerate}
\item (Problem 8.44 pg 320)  A smooth pipe designed to carry crude oil (D = 47 inches, $\rho$= 1.75 $slugs/ft^2$, and $\mu$=$4\times10^{-4}~lbf-s/ft^2$ is to be modeled with a smooth pipe 4 inches in diameter carrying water ($T$=60$^oF$).

If the mean velocity in the prototype is to be 2 $ft/s$, what should be the mean velocity of the water in the model to ensure dynamically similar conditions?

\item (Problem 8.66 pg 322) Flow around a bridge pier is studied using a model at $\frac{1}{12}$ scale.   When the velocity in the model is 0.9 $m/s$, the standing wave at the pier nose is observed to be 2.5 $cm$ in height.  What are the corresponding values of velocity and wave height in the prototype?
\clearpage
(Problem 8.66 pg 322) (Continued)
\end{enumerate}
\end{document}  