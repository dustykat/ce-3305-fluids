\documentclass[12pt]{article}
\usepackage{geometry}                % See geometry.pdf to learn the layout options. There are lots.
\geometry{letterpaper}                   % ... or a4paper or a5paper or ... 
%\geometry{landscape}                % Activate for for rotated page geometry
\usepackage[parfill]{parskip}    % Activate to begin paragraphs with an empty line rather than an indent
\usepackage{daves,fancyhdr,natbib,graphicx,dcolumn,amsmath,lastpage,url}
\usepackage{amsmath,amssymb,epstopdf,longtable}
\usepackage{paralist}  % need to properly formulate standard answer blocks
\usepackage[final]{pdfpages}
\DeclareGraphicsRule{.tif}{png}{.png}{`convert #1 `dirname #1`/`basename #1 .tif`.png}
\pagestyle{fancy}
\lhead{CE 3305 Fluid Mechanics; Exercise Set 1}
\rhead{Name:\_\_\_\_\_\_\_\_\_\_\_\_\_\_\_\_\_\_\_\_\_\_\_\_\_\_\_\_\_\_\_\_\_\_}
\lfoot{REVISION A}
\cfoot{}
\rfoot{Page \thepage\ of \pageref{LastPage}}
\renewcommand\headrulewidth{0pt}
%%%%%%%%%%%%%%%%%%%%%%%%%%%%%%%%%%%%
\begin{document}
%%%%%%%%%%%%%%%%%%%%%%%%%%%%%%%%%%%
\begingroup
\begin{center}
{\textbf{{ CE 3305 Engineering Fluid Mechanics} \\ Exercise Set 1 \\ Summer 2018 -- GERMANY} }
\end{center}
\endgroup
\begingroup
~\newline
\textbf{Purpose} : Apply the ideal gas law under isothermal conditions.  Perform analysis in absolute and gage pressures. \\
\textbf{Assessment Criteria} : Completion, results plausible, format correct, \textbf{R} script shown\\~\\
\textbf{Exercises}:

\begin{enumerate}
\item (Problem 1.34 pg 26)
Natural gas is stored in a spherical tank at a temperature of 10$^o$C.  
At a given initial time, the pressure in the tank is 100 kPa--gage, and the atmospheric pressure is 100 kPa--absolute.  
Some time later, after more gas has been compressed into the tank, the pressure in the tank is 200 kPa--gage, and the temperature is still 10$^o$C.
What is the mass ratio of gas in the tank when $p$ =  200 kPa--gage, to when the pressure was 100 kPa--gage?
\item Write a \textbf{R} script to handle the computations and show the results of the script.
\end{enumerate}


\end{document}  